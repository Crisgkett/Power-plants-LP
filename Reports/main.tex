\documentclass{article}

\usepackage[utf8]{inputenc}
\usepackage{graphicx} % Comandos para manejar imágenes
\usepackage{amsmath}


\graphicspath{ {./images/} } % Carpeta de imágenes

\setlength{\parskip}{2mm} % Espaciado

\usepackage[utf8]{inputenc}
\usepackage{geometry}
    \geometry{left=3cm,right=2cm,top=2cm,bottom=2cm}
%
%\usepackage[spanish]{babel}
%
\usepackage[fixlanguage]{babelbib}
    \bibliographystyle{babunsrt}
%

\usepackage{floatrow}
\floatsetup[table]{style=plaintop}

\usepackage{url}

\usepackage[top=2cm, bottom=2.5cm, right=3 cm, left=3 cm]{geometry} % margenes

\usepackage{parskip} % Sangria


\title{Report One EII 879 \textit{Optimización Complementaria}}
\author{Cristóbal Galleguillos Ketterer^{1}$\\
\small{$^{1}$Industrial PhD Program}\\
\small{Pontificia Universidad Católica de Valparaíso}\\
\small{cristobal.galleguillos@pucv.cl}}
\date{\small{\today}}

\begin{document}

\maketitle

\section{Introduction}

A brief approximation to the optimization in thermal power plants

The energy system is the most complex system developed by humanity, in it, a serial of scenarios of transformations of energy,  electricity generation, and despatch to the end usuaries occur simultaneously.

In the beginning, the industrialization of the world needed a little bit of energy, mainly as a natural resource without major transformation (solar radiation for cooking, wind for pump water on a farm).

Successively, increases in production and transportation led to the development of the technologies use of energy stored in fossil fuels, which allowed the development of the so-called industrial revolution.

Since that date, the impacts of the use of fossil fuels have affected the quality of life of people who live nearby or work in the generating plants, in addition to other less visible effects on the scale of daily living, such as global warming.

As a technological, economic, social, and environmental phenomenon, the generation of electric power is the focus of research in much of the engineering sciences.

In this work report, a very brief introduction to the optimization of energy production in thermal power plants is presented.


\section{Around the problem}

\subsection{The system}

The power system is a balanced and simultaneous set of machines and facilities destined to provide the energy electric to industrial and domestic users.

This system is an unperfect market whit different conditions for competence and prices, generally is a spot market. But the demand must be satisfied all the time.

\subsection{The fossil thermal technologies}

Thermal technologies are the process of transformation of the energy from a fossil combustible burned in a boiler, to produce water vapor to spin a turbine.

This is easy use, known and commercial technologies, the storage (carbon pile, fuel tanks, liquified gas tanks) provider high assurance to the producer to meet your commitments.

However the combustion process is highly inefficient, a lot of energy in heat form is lost at the scape gasses. In additión (and the most important) the combustion processes generated greenhouse gases like the C=2 and other pollutants.

\subsection{The regulatión options (some)}

The following regulation alternatives (and the detail of the problem) are described in the Book \cite{soroudi_power_2017} and correspond to the models 3.4, 3.6, 3.7, 3.8 for this book.

\begin{itemize}
\item Economical Dispatch without emission taxes or limits (ED). 
\item Minimizing the total emission (END).
\item Incorporating taxes to the emissions (Penalty), and 
\item Limiting the emissions (Elimit). 
\end{itemize}

These alternatives generated a four-set of linear programming problems.

\subsection{Mathematical issues}

The model is a Quadratic Constrained Programming (QCP) problem, whit capacities constraints (generation and/or emission max and min limits). The objective function es minimize the cost (ED, Penalty, Elimit) or the emissions (END)

The results show below as obtained whit the academic version of a commercial solver.

\section{Results}

The principal results are shown as the follow table:

\begin{table}[H]
\begin{tabular}{l|l|l|l|l|}
\cline{2-5}
                                 & ED      & End     & Penalty & Elimit  \\ \hline
\multicolumn{1}{|l|}{gen 1}      & 102,844 & 71,622  & 103     & 94      \\ \hline
\multicolumn{1}{|l|}{gen 2}      & 90,000  & 90,000  & 90      & 90      \\ \hline
\multicolumn{1}{|l|}{gen 3}      & 76,730  & 68,000  & 73      & 68      \\ \hline
\multicolumn{1}{|l|}{gen 4}      & 77,425  & 129,763 & 81      & 95      \\ \hline
\multicolumn{1}{|l|}{gen 5}      & 53,000  & 40,615  & 53      & 53      \\ \hline
\multicolumn{1}{|l|}{Total cost $USD$} & 131.385 & 148.615 & 140.925 & 133.112 \\ \hline
\multicolumn{1}{|l|}{Production $MW$} & 400     & 400     & 400     & 400     \\ \hline
\multicolumn{1}{|l|}{Unitary cost $\frac{USD}{MW}$}     & 328     & 372     & 352     & 333     \\ \hline
\end{tabular}
\label{tabla1}
\end{table}

The numerical and hardware aspects ist not relevant in this fase of the work

\section{Coments}

The use of optimization tools can help to better understand the uniqueness of the markets and be of use to the various players.

The regulator can adequately establish incentives and penalties based on the stability of the electricity market, public health, and the environment.

The producer can adapt its technologies in a cost-efficient way, making the most of the first energy resources, by selecting the alternatives that present the highest profitability.

Customers can make sustainable decisions if they have complete information on the traceability of their energy consumption.

In the particular case of the fictitious exercises proposed \cite{soroudi_power_2017} , we will consider the regulator as the entity that observes the results of the emissions of the different scenarios, and the producer, who analyzes the marking according to unit costs.

It is consistent with the intuition, that the deregulated alternative (ED) represents the lowest generation cost for the producer ($328 USD$) and at the same time generates a greater social impact ($96 tons$ of emissions).

On the other hand, the END model is that it presents greater social benefits ($87 tons$) but in practice, it does not seem very feasible for the regulator to impose a production restriction in order to minimize emissions.

Regulating, either by limiting emissions (Elimit) or penalizing them (Penalty) with a tax, in practice is a technically more feasible, verifiable, and auditable solution. Within that, and in the example observed, the limitation (Elimit) presents a scenario that adequately balances the cost and the Ambiental impact.

The conclusions presented above only reflect the analysis of the fictitious problem. For a more sophisticated analysis, I would consider sensitivity analysis, considering that there are quadratic functions.

  \nocite{*}
    \bibliography{src/ref}

\end{document}

